\documentclass{article}

\usepackage{calc}
\usepackage[framemethod=tikz]{mdframed}
\usepackage{tikz}
\usetikzlibrary{decorations.markings}
\usepackage{environ,varwidth}

\mdfdefinestyle{tight}{roundcorner=1.5pt,innerleftmargin=0.7pt,innerrightmargin=0.7pt,innertopmargin=1pt,innerbottommargin=1pt,backgroundcolor=gray!20}

\newsavebox\MyTempBox
\NewEnviron{tight}{%
\savebox\MyTempBox{%
\begin{varwidth}{\linewidth}
\BODY
\end{varwidth}}%
\begin{minipage}{\dimexpr\wd\MyTempBox+2.5pt\relax}
\begin{mdframed}[style=tight,userdefinedwidth=\dimexpr\wd\MyTempBox+2.5pt\relax]
\BODY
\end{mdframed}%
\end{minipage}
}%

% \newcommand\sno[1]{\begin{tight}\scriptsize\bf #1\end{tight}}


\newcommand{\fx}[2]{\texttt{=#1}(#2)}
\newcommand{\fxSTR}[1]{\texttt{"#1"}}
\newcommand{\fxref}[2]{\texttt{#1#2}}
\newcommand{\fxrange}[4]{\fxref{#1}{#2}:\fxref{#3}{#4}}



\newcounter{sno}
\setcounter{sno}{0}
% \newcommand\sno{\stepcounter{sno}$_$}

\newif\ifsnocaption
% \snocaptiontrue
\snocaptionfalse

\newcommand\sno[1]{\stepcounter{sno}\begin{tight}\scriptsize\bf {\thesno} {\ifsnocaption #1 \else {} \fi}\end{tight}}
\newcommand\snp[1]{\sno{1}} % i sometimes typo p instead of o

% \newcommand\sno[1]{}

\newif\ifgood
\goodfalse

\newif\ifbad
\badfalse


\newif\ifexplode
\explodefalse
% \explodetrue

\newcommand\good[1]{\ifgood {\color{olive} #1} \else {#1} \fi}

\newcommand\bad[1]{\ifbad {\color{gray} #1} \else {} \fi}

\newcommand\explode{\ifexplode {\clearpage} \else {} \fi}

\newif\ifInlineNotes
\InlineNotesfalse
% \InlineNotestrue

\newcommand\inlineNote[1]{\ifInlineNotes \hl{\tt #1} \else {} \fi}


\usepackage{glossaries}
 
\makeglossaries
 
\newglossaryentry{PPL}
{
    name=PPL,
    description={Is a mark up language specially suited 
    for scientific documents}
}

\newglossaryentry{bidi}
{
    name=Binomial distribution,
    description={TBD}
}
\newglossaryentry{latent}
{
    name=Latent variable,
    description={TBD}
}
\newglossaryentry{de}
{
    name=Data enthusiast,
    description={TBD}
}
\newglossaryentry{msg}
{
    name=Message-passing,
    description={TBD}
}
\newglossaryentry{bayesian model}
{
    name=Bayesian model,
    description={TBD}
}
\newglossaryentry{alphabeta}
{
    name=Alpha and Beta distribution,
    description={TBD}
}
\newglossaryentry{fgraph}
{
    name=Factor graphs,
    description={TBD}
}
\newglossaryentry{prior}
{
    name=Prior and posterior,
    description={TBD}
}
\newglossaryentry{univinf}
{
    name=Universal inference,
    description={TBD}
}


\usepackage[utf8]{inputenc}
\usepackage{enumitem}
\makeatletter
\renewcommand\paragraph{\@startsection{paragraph}{4}{\z@}%
                                    {3.25ex \@plus1ex \@minus.2ex}%
                                    {-1em}%
                                    {\normalfont\normalsize}}
\makeatother

\title{Empowering Spreadsheet Users with Probabilistic Programming, Lecture 1}
\author{Lecturer: Andrew D. Gordon}
\date{June 24 2019}


\newtheorem{example}{Example}[section]

\begin{document}

\maketitle

\section{Probabilistic Programming}
\sno{}
Probabilistic programming is an inter-paradigm style of programming in which users
\begin{enumerate}
\item invokes randomness for probabilistic behavior,
\item add {\bf constraints} to condition on observed data, and
\item indicate which variables' distributions are to be inferred.
\end{enumerate}

\sno{}
This approach is both easier and more expressive than writing code for probabilistic behavior from scratch making it a popular choice for both academic and commercial \gls{PPL} (Probabilistic Programming Language) technologies (BUGS, Figaro, pcc, Church, Dimple, Hansei, STAN, Infer.NET/Fun, R2, Factorie, BLOG, ProbLog, Alchemy, Venture, Anglican, Wolfe, Hakaru, Edward…). The semantics of PPLs was pioneered by Kozen, Giry, Jones/Plotkin, Panangaden et al, Ramsey/Pfeffer.

\input{data-enthused}

\section{The Big 4 PPLs}
\begin{enumerate}[label=\Roman*]
    \item BUGS (1989)
    \begin{enumerate}[label=\roman*]
        \item Pioneering work in the field
        \item Introduced the idea of expressing probabilistic models decoratively
    \end{enumerate}
    \item Infer.net (2004) \cite{Minka:18:InferNET}
    \begin{enumerate}[label=\roman*]
        \item Uses a message passing infrastructure
        \item Based on BUGS
    \end{enumerate}
    \item Church (2007)
    \begin{enumerate}[label=\roman*]
        \item Turing complete (general purpose)
        \item A superset of lisp
    \end{enumerate}
    \item Stan (2017)
    \begin{enumerate}[label=\roman*]
        \item Started as a dialect of BUGS
    \end{enumerate}
\end{enumerate}


\begin{example}
Two coins are flipped in secret. It is guaranteed that they are not BOTH heads. What is the probability that the first coin is heads?
\end{example}

This is a \LaTeX{} test.
And it seems like it's working.

\bibliography{prob-prog-refs}{}
\bibliographystyle{plain}

\clearpage
 
\glsaddall
\printglossaries

\end{document}